\documentclass[compress]{beamer}

\usepackage{ctex}

% beamer
\PassOptionsToPackage{subsection=false}{beamerouterthememiniframes}
\setbeamerfont{title}{size=\huge}
\setbeamertemplate{itemize item}{$\bullet$}
\setbeamertemplate{bibliography item}[text]
%gets rid of bottom navigation bars
\setbeamertemplate{footline}[page number]{}
\setbeamertemplate{navigation symbols}{}
\setbeamertemplate{frametitle}{
    \begin{center}
        \insertframetitle
        \par
    \end{center}
}

\usepackage{etoolbox}
\makeatletter
\patchcmd{\slideentry}{\advance\beamer@xpos by1\relax}{}{}{}
\def\beamer@subsectionentry#1#2#3#4#5{\advance\beamer@xpos by1\relax}%
\makeatother

% toc
\AtBeginSection[]{
    \begin{frame}{Outline}
        \begin{columns}
            \begin{column}{0.3\textwidth}
            \end{column}
            \begin{column}{0.7\textwidth}
                \tableofcontents[currentsection, hideallsubsections] 
            \end{column}
        \end{columns}
    \end{frame} 
}

\usepackage{minted}

%http://tex.stackexchange.com/q/113168
\makeatletter
% avoid space tokens since we're in horizontal mode
\renewcommand\mint[3][]{%
  \DefineShortVerb{#3}%
  \minted@resetoptions
  \setkeys{minted@opt}{#1}%
  \SaveVerb[aftersave={%
    \UndefineShortVerb{#3}%
    \minted@savecode{\FV@SV@minted@verb}%
    \minted@pygmentize{#2}%
    \DeleteFile{\jobname.pyg}}]{minted@verb}#3}
\renewcommand\minted@savecode[1]{%
  \immediate\openout\minted@code\jobname.pyg\relax
  \immediate\write\minted@code{#1}%
  \immediate\closeout\minted@code}
\renewcommand\minted@pygmentize[2][\jobname.pyg]{%
  \def\minted@cmd{pygmentize -l #2 -f latex -F tokenmerge
    \minted@opt{gobble} \minted@opt{texcl} \minted@opt{mathescape}
    \minted@opt{linenos} -P "verboptions=\minted@opt{extra}"
    -o \jobname.out.pyg #1}%
  \immediate\write18{\minted@cmd}%
  \ifthenelse{\equal{\minted@opt@bgcolor}{}}%
   {}%
   {\begin{minted@colorbg}{\minted@opt@bgcolor}}%
  \input{\jobname.out.pyg}%
  \ifthenelse{\equal{\minted@opt@bgcolor}{}}%
   {}%
   {\end{minted@colorbg}}%
  \DeleteFile{\jobname.out.pyg}}
\makeatother
\RecustomVerbatimEnvironment{Verbatim}{BVerbatim}{}


\usepackage{etoolbox}
\makeatletter
\patchcmd{\slideentry}{\advance\beamer@xpos by1\relax}{}{}{}
\def\beamer@subsectionentry#1#2#3#4#5{\advance\beamer@xpos by1\relax}%
\makeatother

% toc
\AtBeginSection[]{
    \begin{frame}{Outline}
        \begin{columns}
            \begin{column}{0.3\textwidth}
            \end{column}
            \begin{column}{0.7\textwidth}
                \tableofcontents[currentsection, hideallsubsections] 
            \end{column}
        \end{columns}
    \end{frame} 
}

%footnote
\usepackage{url}
\newcommand{\footurl}[1]{\footnote{\url{#1}}}

% figure
\usepackage{graphicx}
\usepackage{adjustbox}
\usepackage{caption}
\captionsetup{labelformat=empty,labelsep=none}
\usepackage{subcaption}
\usepackage[multidot]{grffile}

% cross over
\usepackage[normalem]{ulem}

% shorthand
\newcommand{\fname}[1]{\texttt{#1}}

\title{Modern C++ Coding}
\subtitle{abstraction}
\author{answeror@gmail.com}
\date{2014-02-21}

\begin{document}

\frame{\titlepage}

\begin{frame}{Slides}
    \url{https://github.com/Answeror/moderncppcoding}
\end{frame}


\begin{frame}{一切皆为抽象}
    \begin{itemize}[<+->]
        \item 编程语言和数学系统的核心皆为抽象.
        \item 变量, 函数, OOP都是针对各自领域问题的抽象手段.
        \item OOP经常在不合适的领域被\textcolor{red}{极度}滥用.
    \end{itemize}
\end{frame}

\section{函数}

\begin{frame}{函数}
    \begin{itemize}[<+->]
        \item 在科学计算领域, 函数是最好的抽象手段.\footnote{个人认为}
        \item (纯)函数\footnote{pure function}易于描述(输入, 输出, 前条件, 后条件).
        \item (纯)函数易于测试.
    \end{itemize}
\end{frame}

\begin{frame}{编码过程}
    \begin{enumerate}[<+->]
        \item 首先明确问题(函数)的输入输出.
        \item 构造数据, 写针对函数的单元测试.
        \item 分解主函数为多个子函数, 写子函数的单元测试.
        \item 子函数之间通过函数参数传递状态.
        \item 若函数参数列表太长\footnote{大于7, 参见代码大全.}, 则将共享参数的函数打包成类.
    \end{enumerate}
\end{frame}

\section{模板(template)}

\begin{frame}{模板(template)}
    \begin{itemize}[<+->]
        \item duck typing: If it looks like a duck and quacks like a duck, it's a duck.
        \item C++模板提供了编译期的duck typing手段.
    \end{itemize}
\end{frame}

\begin{frame}{duck typing in Python}
    \inputminted[fontsize=\footnotesize]{python}{mean.py}
\end{frame}

\begin{frame}{duck typing in C++}
    \inputminted[fontsize=\footnotesize]{cpp}{mean.cpp}
\end{frame}

\begin{frame}{duck typing in C++}
    \inputminted[fontsize=\footnotesize]{cpp}{use_mean.cpp}
\end{frame}

\newcommand{\typeerasure}{类型擦除(type erasure)}
\section{\typeerasure}

\begin{frame}{\typeerasure}
    \begin{itemize}[<+->]
        \item 有时候模板带来的编译负担太重\footnote{还有源码保密等因素}, 需要借助虚函数减少代码生成.
        \item 但是很多时候用户无法控制库的类层次结构.
    \end{itemize}
\end{frame}

\begin{frame}{\typeerasure}
    将具有一个共通接口的形形色色的类型变成具有相同接口的``一个''类型\footnote{``C++模板元编程''9.7.5节.}.
\end{frame}

\begin{frame}
    假设机器学习库ml提供了QDA, SVM等具有相同接口的分类器(\mint{cpp}/fit/, \mint{cpp}/predict/):

    \vspace{1em}

    \inputminted[fontsize=\footnotesize]{cpp}{qda.hpp}
\end{frame}

\begin{frame}
    另一个机器学习库yaml提供了AdaBoost, CART等分类器, 但是接口与ml不同:

    \begin{itemize}
        \item \mint{cpp}/fit/ $\to$ \mint{cpp}/train/
        \item \mint{cpp}/predict/ $\to$ \mint{cpp}/transform/
    \end{itemize}
\end{frame}

\begin{frame}
    现在我们需要写一个函数对分类器的性能做评估:

    \vspace{1em}

    \inputminted[fontsize=\footnotesize]{cpp}{core_fwd.hpp}
\end{frame}

\begin{frame}
    \begin{center}
        How to write \textcolor{red}{classifier}?
    \end{center}
\end{frame}

\begin{frame}{the core of type erasure}
    ml和yaml库内部的类之间可能并不存在继承体系. 我们需要一个额外的基类.

    \vspace{1em}

    \inputminted[fontsize=\footnotesize]{cpp}{clf.hpp}
\end{frame}

\begin{frame}{the core of type erasure}
    利用模板和虚函数包装具有相同\textcolor{red}{接口}的类, 即``擦除''了真实类型, 对外仅公布\mint{cpp}/classifier/的接口.

    \vspace{1em}

    \inputminted[fontsize=\footnotesize]{cpp}{clf_ml.hpp}
\end{frame}

\begin{frame}{the core of type erasure}
    \inputminted[fontsize=\footnotesize]{cpp}{clf_yaml.hpp}
\end{frame}

\begin{frame}{the core of type erasure}
    \inputminted[fontsize=\footnotesize]{cpp}{use_core_te.hpp}
\end{frame}

\begin{frame}{Boost.TypeErasure}
    \inputminted[fontsize=\footnotesize]{cpp}{boost_te.hpp}
    \vskip5mm
    Boost.TypeErasure提供了组合接口的各种组件. 上面的classifier与之前手工定义的\mint{cpp}/classifier_ml/具有相似的功能.
\end{frame}

\begin{frame}{组合爆炸}
    \begin{itemize}[<+->]
        \item 不同的模块可能需要功能上不完全相同, 却又包含一定重叠的接口. 功能重叠的``类型擦除器''会导致``组合爆炸''.
        \item Boost.TypeErasure提供了一种通过模板定制``类型擦除器''的接口的手段. 类似的还有Boost.Function(已纳入C++11标准), Boost.Range和Boost.Any.
    \end{itemize}
\end{frame}

\begin{frame}{C style optimization}
    \inputminted[fontsize=\footnotesize]{cpp}{optimize.h}
    \vskip5mm
    \pause
    \begin{itemize}[<+->]
        \item 只能传递函数指针
        \item 容易诱使全局变量产生
        \item 难以并行化
        \item 解决办法一般是在f签名中增加表示数据的void指针
    \end{itemize}
\end{frame}

\begin{frame}{OOP style optimization}
    \inputminted[fontsize=\footnotesize]{cpp}{old_optimize.hpp}
    \vskip5mm
    \pause
    \begin{itemize}[<+->]
        \item 侵入式, 要求继承\mint{cpp}/fn/.
        \item 代码长(Keep it simple, stupid!).
    \end{itemize}
\end{frame}

\begin{frame}{template style optimization}
    \inputminted[fontsize=\footnotesize]{cpp}{template_optimize.hpp}
    \vskip5mm
    \pause
    \begin{itemize}[<+->]
        \item 所有实现必须写在头文件里, 编译慢.
        \item 源码泄漏.
    \end{itemize}
\end{frame}

\begin{frame}{modern C++ style optimization}
    \inputminted[fontsize=\footnotesize]{cpp}{modern_optimize.hpp}
\end{frame}

%\section{range}

\begin{frame}
    \begin{center}
        \huge{Thanks}
    \end{center}
\end{frame}

\end{document}
