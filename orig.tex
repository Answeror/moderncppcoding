\documentclass[compress]{beamer}

\usepackage{ctex}

% beamer
\PassOptionsToPackage{subsection=false}{beamerouterthememiniframes}
\usetheme{Berlin}
\usecolortheme{dolphin}
\setbeamerfont{title}{size=\huge}
%\setbeamertemplate{caption}{\insertcaption}
%\setbeamertemplate{caption label separator}{}
\setbeamertemplate{itemize item}{$\bullet$}
%\usefonttheme{structuresmallcapsserif}
%\usepackage{times}
%\usepackage[T1]{fontenc}
\setbeamertemplate{bibliography item}[text]
%gets rid of bottom navigation bars
\setbeamertemplate{footline}[page number]{}
%gets rid of navigation symbols
\setbeamertemplate{navigation symbols}{}
%\beamertemplatenavigationsymbolsempty
%\setbeamertemplate{frametitle}[default][center]
\setbeamertemplate{frametitle}
{
    \begin{center}
        \insertframetitle
        \par
    \end{center}
}

\usepackage{etoolbox}
\makeatletter
\patchcmd{\slideentry}{\advance\beamer@xpos by1\relax}{}{}{}
\def\beamer@subsectionentry#1#2#3#4#5{\advance\beamer@xpos by1\relax}%
\makeatother

% toc
\AtBeginSection[] { 
    \begin{frame} 
        \frametitle{Outline} 
        \tableofcontents[currentsection, hideallsubsections] 
    \end{frame} 
}

%footnote
\usepackage{url}
\newcommand{\footurl}[1]{\footnote{\url{#1}}}

%code
\usepackage{minted}

% figure
\usepackage{graphicx}
\usepackage{adjustbox}
\usepackage{caption}
\captionsetup{labelformat=empty,labelsep=none}
\usepackage{subcaption}
\usepackage[multidot]{grffile}

% cross over
\usepackage[normalem]{ulem}

% shorthand
\newcommand{\fname}[1]{\texttt{#1}}

\title{Re-search v2}
\author{answeror@gmail.com}
\date{2013-10-15}

\begin{document}

\frame{\titlepage}

\begin{frame}{Research = Re-search}
    \begin{figure}
        \centering
        \includegraphics[width=\textwidth]{images/library.jpg}
        \caption{You are not the Laughing Man\ldots}
    \end{figure}
\end{frame}

\section{基本搜索}

\subsection{基本技术}

\begin{frame}{\insertsubsection}
    \begin{itemize}[<+->]
        \item `CTRL` + `F`
        \item \textcolor{red}{Google}\footnote{not 谷歌 or 百度} it! (\url{http://google.com/ncr})
    \end{itemize}
\end{frame}

\subsection{搜索技巧}

\begin{frame}{\insertsubsection}
    \begin{columns}[T]
        \begin{column}{0.5\textwidth}
            \begin{itemize}[<+->]
                \item 搜索指定域名: \\ \vspace{0.1cm} \textbf{奖学金 \textcolor{blue}{site}:grs.zju.edu.cn}
                \item 搜索指定文件类型: \\ \vspace{0.1cm} \textbf{svm \textcolor{blue}{filetype}:m}
                \item 搜索图片:
                \item 更多技巧: \\ \vspace{0.1cm} \url{http://goo.gl/u1ug9}
            \end{itemize}
        \end{column}
        \begin{column}{0.5\textwidth}
            \visible<3>{\includegraphics[width=\textwidth]{images/image-search.png}}
        \end{column}
    \end{columns}
\end{frame}

\subsection{网页历史}

\begin{frame}
    \begin{figure}
        \includegraphics[width=0.9\textwidth]{images/404}
        \caption{并不是所有404都能翻得过去\ldots}
    \end{figure}
\end{frame}

\begin{frame}{\insertsubsection}
    \begin{columns}
        \begin{column}{0.3\textwidth}
        \end{column}
        \begin{column}{0.7\textwidth}
            \begin{itemize}[<+->]
                \item Google Cache
                \item Google Translate
                \item \url{http://archive.org}
            \end{itemize}
        \end{column}
    \end{columns}
\end{frame}

\subsection{Stackoverflow}

\begin{frame}{\insertsubsection}
    \begin{itemize}[<+->]
        \item 高质量的技术问答.
        \item 姊妹站点: math, tex, stats, serverfault, superuser\ldots
    \end{itemize}
\end{frame}

\begin{frame}{\insertsubsection}
    \begin{columns}[T]
        \begin{column}{0.4\textwidth}
            \includegraphics[width=\textwidth]{images/mine}
        \end{column}
        \begin{column}{0.6\textwidth}
            \textbf{小型可穿戴式动作捕获装置}
            \par
            \vspace{1em}
            去年10月10日晚上, 因为第二天需要演示实验室的一个遗留项目, 从stackoverflow上挖出来一个传感器聚合算法.
        \end{column}
    \end{columns}
\end{frame}

\section{文本搜索}

\begin{frame}{Warning}
    本节内容并不适用于所有研究方向. 有些研究方向只需要专心读某几个期刊/会议的文章即可.
\end{frame}

\subsection{WOS}

\begin{frame}[t]{\insertsubsection}
    \begin{itemize}[<+->]
        \item WOS关键词检索

            \only<1>{
                \begin{center}
                    \includegraphics[width=0.7\textwidth]{images/wos.search.png}
                \end{center}
            }
        \item 添加约束条件

            \only<2>{
                \begin{center}
                    \includegraphics[width=0.4\textwidth]{images/wos.refine.png}
                \end{center}
            }
        \item 导出纯文本

            \only<3>{
                \begin{center}
                    \includegraphics[width=0.7\textwidth]{images/wos.export.png}
                \end{center}
            }
    \end{itemize}
\end{frame}

\subsection{HistCite}

\begin{frame}[t]{\insertsubsection}
    \begin{itemize}[<+->]
        \item 导入HistCite(History of Cite)

            \only<1>{
                \begin{center}
                    \includegraphics[width=0.7\textwidth]{images/histcite.png}
                \end{center}
            }
        \item 筛选综述类文献(按\textbf{CR}排序)

            \only<2>{
                \begin{center}
                    \includegraphics[width=0.7\textwidth]{images/histcite.cr.png}

                    \tiny{综述类文献通常带有"review", "survey", "state of art"等字眼.}
                \end{center}
            }
        \item 筛选领域内重要文献(按\textbf{LCS}排序)

            \only<3>{
                \begin{center}
                    \includegraphics[width=0.7\textwidth]{images/histcite.lcs.png}
                \end{center}
            }
        \item \textbf{作图}

            \only<4>{
                \begin{center}
                    \includegraphics[width=0.7\textwidth]{images/histcite.make-graph.png}
                \end{center}
            }
    \end{itemize}
\end{frame}

\begin{frame}{\insertsubsection}
    \begin{figure}
        \centering
        \includegraphics[width=\textwidth]{images/histcite.graph.png}
        \caption{从图中可以看出69是一篇开创性的工作.}
    \end{figure}
\end{frame}

\subsection{pypaper}

\begin{frame}
    \begin{center}
    如果我们要的文章不在WOS怎么办?
    \end{center}
\end{frame}

\begin{frame}{Google Scholar}
    \begin{figure}
        \centering
        \includegraphics[width=0.8\textwidth]{images/scholar.png}
    \end{figure}
\end{frame}

\begin{frame}
    \begin{center}
    但是Google Scholar不提供排序功能.
    \end{center}
\end{frame}

\begin{frame}[fragile]{bibtex}
    \begin{listing}[H]
        \begin{minted}[
            fontsize=\footnotesize,
            frame=lines,
            framesep=2mm
        ]{bibtex}
@Book{abramowitz+stegun,
 author    = "Milton {Abramowitz} and Irene A. {Stegun}",
 title     = "Handbook of Mathematical Functions with
              Formulas, Graphs, and Mathematical Tables",
 publisher = "Dover",
 year      =  1964,
 address   = "New York",
 edition   = "ninth Dover printing, tenth GPO printing"
}
        \end{minted}
    \end{listing}
\end{frame}

\begin{frame}{\insertsubsection}
    Pypaper\footurl{https://github.com/Answeror/pypaper}是一套Python脚本, 用于:

    \begin{itemize}
        \item 自动批量下载IEEE和ACM的bibtex和PDF\footnote{用户手册没写好.}.
        \item 自动根据bibtex批量抓取Google Scholar信息.
    \end{itemize}
\end{frame}

\begin{frame}{\insertsubsection}
    \begin{itemize}[<+->]
        \item 使用tor\footurl{https://www.torproject.org/}动态切换ip.
        \item 使用加权编辑距离\footurl{https://github.com/Answeror/lit\#introduction}从搜索结果中选择最佳匹配.
    \end{itemize}
\end{frame}

\begin{frame}{\insertsubsection}
    \begin{figure}
        \centering
        \includegraphics[width=0.8\textwidth]{images/motion-texture.png}
        \caption{加权编辑距离的应用场景(不同的编辑操作, 乃至操作的位置(词首/词尾)应具有不同的权重)}
    \end{figure}
\end{frame}

\begin{frame}[fragile]{\insertsubsection}
    \begin{listing}[H]
        \begin{minted}[
            fontsize=\footnotesize,
            frame=lines,
            framesep=2mm
        ]{bat}
python scripts\google_fetch.py -h
python scripts\google_query.py -h
        \end{minted}
        \caption{usage}
    \end{listing}

    \begin{listing}[H]
        \begin{minted}[
            fontsize=\footnotesize,
            frame=lines,
            framesep=2mm
        ]{bat}
python scripts\google_fetch.py -d all.db all.bib
python scripts\google_query.py -d all.db -o out.txt
python scripts\google_query.py -d all.db -o out.txt review.bib
        \end{minted}
        \caption{pypaper example}
    \end{listing}
\end{frame}

\begin{frame}[fragile]{\insertsubsection}
    \begin{listing}[H]
        \begin{minted}[
            fontsize=\footnotesize,
            frame=lines,
            framesep=2mm
        ]{jinja}
01436 Human motion analysis: A review
00996 Anomaly detection: A survey
00441 Machine recognition of human activities: A survey
00269 A survey of mobile phone sensing
00138 Assessment of physical activity: a critical appraisal
00118 Human motion tracking for rehabilitation—A survey
00107 Activity identification using body-mounted sensors—a ...
00052 Situation identification techniques in pervasive comp...
00031 Activity recognition using inertial sensing for healt...
00014 Advances in physical activity monitoring and lifestyl...
00013 The use of wearable inertial motion sensors in human ...
00002 Physical activity monitoring by use of accelerometer-...
00002 A survey on human activity recognition using wearable...
        \end{minted}
        \caption{out.txt}
    \end{listing}
\end{frame}

\subsection{后续工作}

\begin{frame}{\insertsubsection}
    \begin{itemize}[<+->]
        \item 看看前面整理出来的重要文章一般都发在哪里, 把相应的期刊/会议拿到WOS中搜一搜, 加入到HistCite里分析.
        \item 直接去作者主页看他的publication list.
    \end{itemize}
\end{frame}

\subsection{文献搜索/管理(Mendeley)}

\begin{frame}{\insertsubsection}
    \begin{columns}
        \begin{column}{0.3\textwidth}
        \end{column}
        \begin{column}{0.7\textwidth}
            萝卜白菜, 各有所爱\ldots

            \pause

            \begin{itemize}[<+->]
                \item PDF内容搜索
                \item 层级标签
                \item 跨平台同步
                \item bibtex导出
                \item \sout{Word插件}\footnote{latex please}
            \end{itemize}
        \end{column}
    \end{columns}
\end{frame}

\begin{frame}{\insertsubsection}
    \begin{center}
        不要用mendeley记笔记. Mendeley的中文搜索极差.
    \end{center}
\end{frame}

\subsection{Evernote}

\begin{frame}
    \begin{center}
        \huge{\alert{不可搜索的笔记等于没记.}}
    \end{center}
\end{frame}

\begin{frame}{\insertsubsection}
    \begin{figure}
        \centering
        \includegraphics[width=0.8\textwidth]{images/evernote}
        \caption{集中存放所有笔记}
    \end{figure}
\end{frame}

\begin{frame}{\insertsubsection}
    \begin{figure}
        \centering
        \includegraphics[width=0.8\textwidth]{images/tagging}
        \caption{做好标签}
    \end{figure}
\end{frame}

\section{本地搜索}

\subsection{Everything}

\begin{frame}{\insertsubsection}
    \begin{figure}
        \centering
        \includegraphics[width=\textwidth]{images/everything.png}
        \caption{文件命名: lowercase-and-dash.dot-sep-logical-part.ext}
    \end{figure}
\end{frame}

\subsection{gn}

\begin{frame}{\insertsubsection}
    Gn\footurl{http://file.answeror.com/gn-0.1.0-win32.msi}是一个已打包的Python程序, 用于重命名文件. 使用时将(若干)目标文件拖动到程序图标上即可.

    \begin{center}
        C++ Coding Standard.pdf $\to$ cpp-coding-standard.pdf
    \end{center}
\end{frame}

\begin{frame}[fragile]{配置文件}
    \begin{listing}[H]
        \begin{minted}[
            fontsize=\footnotesize,
            frame=lines,
            framesep=2mm
        ]{jinja}
[cC]\+\+;cpp
[cC]#;csharp
        \end{minted}
        \caption{$\sim$/.gn}
    \end{listing}
\end{frame}

\subsection{lit}

\begin{frame}{F**k ALT+TAB \& WIN+R}
    \begin{center}
        窗口切换和程序执行应该也是\textcolor{red}{可搜索的}.
        \pause
        \par
        Launcher, Switcher, Switcheroo\ldots\footnote{\url{http://superuser.com/q/53573/233118}}
        \pause
        \par
        \vspace{1em}
        lit\footnote{\url{https://github.com/Answeror/lit}}
    \end{center}
\end{frame}

\begin{frame}{\insertsubsection}
    \begin{figure}
        \centering
        \includegraphics[trim=10cm 4cm 10cm 9cm,clip,scale=0.5]{images/lit.go.png}
        \caption{窗口切换}
    \end{figure}
\end{frame}

\begin{frame}{\insertsubsection}
    \begin{figure}
        \centering
        \includegraphics[trim=10cm 4cm 10cm 9cm,clip,scale=0.5]{images/lit.run.png}
        \caption{程序运行}
    \end{figure}
\end{frame}

\begin{frame}{\insertsubsection}
    \begin{figure}
        \centering
        \includegraphics[scale=0.5]{images/noot}
        \caption{加权编辑距离匹配}
    \end{figure}
\end{frame}

\section{}

\begin{frame}
    \begin{center}
        \huge{Thanks}
    \end{center}
\end{frame}

\end{document}
