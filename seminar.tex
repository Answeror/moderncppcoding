\documentclass[compress]{beamer}

\usepackage{ctex}

% beamer
\PassOptionsToPackage{subsection=false}{beamerouterthememiniframes}
\setbeamerfont{title}{size=\huge}
\setbeamertemplate{itemize item}{$\bullet$}
\setbeamertemplate{bibliography item}[text]
%gets rid of bottom navigation bars
\setbeamertemplate{footline}[page number]{}
\setbeamertemplate{navigation symbols}{}
\setbeamertemplate{frametitle}{
    \begin{center}
        \insertframetitle
        \par
    \end{center}
}

\usepackage{etoolbox}
\makeatletter
\patchcmd{\slideentry}{\advance\beamer@xpos by1\relax}{}{}{}
\def\beamer@subsectionentry#1#2#3#4#5{\advance\beamer@xpos by1\relax}%
\makeatother

% toc
\AtBeginSection[]{
    \begin{frame}{Outline}
        \begin{columns}
            \begin{column}{0.4\textwidth}
            \end{column}
            \begin{column}{0.6\textwidth}
                \tableofcontents[currentsection, hideallsubsections] 
            \end{column}
        \end{columns}
    \end{frame} 
}

%footnote
\usepackage{url}
\newcommand{\footurl}[1]{\footnote{\url{#1}}}

%code
\usepackage{minted}

%http://tex.stackexchange.com/q/113168
\makeatletter
% avoid space tokens since we're in horizontal mode
\renewcommand\mint[3][]{%
  \DefineShortVerb{#3}%
  \minted@resetoptions
  \setkeys{minted@opt}{#1}%
  \SaveVerb[aftersave={%
    \UndefineShortVerb{#3}%
    \minted@savecode{\FV@SV@minted@verb}%
    \minted@pygmentize{#2}%
    \DeleteFile{\jobname.pyg}}]{minted@verb}#3}
\renewcommand\minted@savecode[1]{%
  \immediate\openout\minted@code\jobname.pyg\relax
  \immediate\write\minted@code{#1}%
  \immediate\closeout\minted@code}
\renewcommand\minted@pygmentize[2][\jobname.pyg]{%
  \def\minted@cmd{pygmentize -l #2 -f latex -F tokenmerge
    \minted@opt{gobble} \minted@opt{texcl} \minted@opt{mathescape}
    \minted@opt{linenos} -P "verboptions=\minted@opt{extra}"
    -o \jobname.out.pyg #1}%
  \immediate\write18{\minted@cmd}%
  \ifthenelse{\equal{\minted@opt@bgcolor}{}}%
   {}%
   {\begin{minted@colorbg}{\minted@opt@bgcolor}}%
  \input{\jobname.out.pyg}%
  \ifthenelse{\equal{\minted@opt@bgcolor}{}}%
   {}%
   {\end{minted@colorbg}}%
  \DeleteFile{\jobname.out.pyg}}
\makeatother
\RecustomVerbatimEnvironment{Verbatim}{BVerbatim}{}


% table
\usepackage{tabu}
\usepackage{booktabs}
\newcolumntype{R}{>{\raggedleft\arraybackslash}X}
\newcommand{\ra}[1]{\renewcommand{\arraystretch}{#1}}
% This is needed because raggedright in table elements redefines \\:
\newcommand{\PreserveBackslash}[1]{\let\temp=\\#1\let\\=\temp}
\let\PBS=\PreserveBackslash
\usepackage[normalem]{ulem}
\newcommand{\textsubscr}[1]{\ensuremath{_{\scriptsize\textrm{#1}}}}

% figure
\usepackage{graphicx}
\usepackage{adjustbox}
\usepackage{caption}
\captionsetup{labelformat=empty,labelsep=none}
\usepackage{subcaption}
\usepackage[multidot]{grffile}

% cross over
\usepackage[normalem]{ulem}

% shorthand
\newcommand{\fname}[1]{\texttt{#1}}

\title{Seminar}
\author{answeror@gmail.com}
\date{2013-11-30}

\begin{document}

\frame{\titlepage}

\section{上阶段工作}

\subsection{工作目标}

\begin{frame}{工作目标}
    \begin{center}
        \ra{1.3}
        \begin{tabu} to \textwidth {lR}
            \toprule
            目标 & 预期完成时间 \\
            \midrule
            PAMAP2上的性能测试 & 2013-09-13 \\
            在线算法的C++实现 & 2013-09-13 \\
            总控流程的PC实现 & 2013-09-20 \\
            安卓移植 & 2013-09-27 \\
            基于部位方法的proposal & 2013-10-11 \\
            运动传感和表面肌电数据库重构 & 2013-11-01 \\
            \bottomrule
        \end{tabu}
    \end{center}
\end{frame}

\subsection{完成情况}

\begin{frame}{完成情况}
    \begin{enumerate}[1)]
        \item 完成走/跑/静止/驾驶行为识别测试和QDA版本移植.
        \item 完成摔倒检测测试.
        \item 完成驾驶人和驾驶行为识别初步测试.
        \item 移植了numpy的基本功能到C++(numcpp).
        \item 特征提取库在线处理, bug修复.
    \end{enumerate}
\end{frame}

\begin{frame}{项目延期原因}
    \begin{enumerate}[1)]
        \item 原始数据处理问题.
        \item androface提交数: 138(2013-01-27至2013-08-21)
        \item androact提交数: 634(2013-08-24至2013-11-29)
    \end{enumerate}
\end{frame}

\section{下阶段计划}

\begin{frame}{工作目标}
    \begin{center}
        \ra{1.3}
        \begin{tabu} to \textwidth {lR}
            \toprule
            目标 & 预期完成时间 \\
            \midrule
            完成驾驶数据处理和驾驶人识别 & 2013-12-07 \\
            完成摔倒检测移植 & 2013-12-07 \\
            完成跑步计数整合 & 2013-12-07 \\
            完成驾驶人识别移植 & 2013-12-14 \\
            浙江省公益项目申请 & 2013-12-18 \\
            驾驶人识别的proposal & ? \\
            基于部位方法的proposal & ? \\
            运动传感和表面肌电数据库重构 & 寒假前 \\
            \bottomrule
        \end{tabu}
    \end{center}
\end{frame}

\section{心得}

\begin{frame}{心得}
    \begin{enumerate}[1)]
        \item 建议所有同学从现在开始使用Python.
        \item 原始数据处理是最关键的步骤, 其文档和测试应最为详细.
    \end{enumerate}
\end{frame}

\end{document}
